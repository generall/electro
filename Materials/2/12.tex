\subsubsection{Фотодиод}

Структура фотодиода не отличается от структуры обычного диода. При облучении полупроводника солнечным светом генерируются концентрация носителей заряда. Они диффундируют в нейтральную область и в область перехода, где неосновные пролетают через переход благодаря полю, а основные задерживаются. Этот ток есть фототок, который способствует уменьшению потенциального барьера создавая фотоЭДС.

Если фотодиод подключен к резистору, то ток в цепи определяется следующим уравнением:
\begin{equation}
I = I_{\Phi} - I_0 (e^{\frac{IR}{\phi_t}}-1)
\end{equation}

Фотодиоды характеризуются спектральной характеристикой, которая отвечает за количество генерируемого фототока
\begin{equation}
I_{\Phi} = S \Phi
\end{equation}
S зависит как от частоты излучения, так и от частоты его интенсивности (мерцания). Определяется материалом и примесями.