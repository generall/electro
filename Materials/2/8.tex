\subsubsection{Стабилитрон. ВАХ %какой стабилитрон
 Система параметров. Применение.}


Стабилитрон - это полупроводниковый прибор, являющийся аналогом газоразрядного стабилитрона. Основное его свойство - резкое увеличение крутизны вольт-амперной характеристики при определенном значении напряжения. ВАХ стабилитрона представлена на рисунке \ref{pic:VAHp}.


Достигается такая крутизна характеристики обычно за счет туннельного или лавинного пробоя (тепловой пробой обычно все портит).
Рассмотрим эти явления:\\
\textbf{Туннельный пробой}\\
Из квантово-механической магии известно, что электрон может преодолевать потенциальный барьер, если его ширина достаточно мала для этого (но не высота).
Вероятность данного события оценивается следующей формулой:
\begin{equation}
P = e^{-10^{-8} d \sqrt{\Phi}}
\end{equation}
Где d - ширина, а $\Phi$ - высота барьера.\\

В случае полупроводника, высота барьера - это ширина запрещенной зоны, а ширина - расстояние между противостоящими зонами.\\
Ширина d оценивается следующим соотношением: $d = l (\phi_z/U)$, отсюда понятно, что вероятность туннелирования тем больше, чем больше напряжение (из-за минуса в экспоненте).
Тогда туннельный ток определяется умножением вероятности туннельного эффекта на некоторые параметры валентной зоны (не уточняется, но наверняка что-то с концентрацией) и на напряжение.
\begin{equation}
j = U A e^{-\frac{\phi_z^{3/2}}{E}}
\end{equation} 
\textbf{Лавинный пробой}\\
Лавинный пробой возникает, если электрон в процессе свободного полета успевает набрать достаточную энергию для выбивания электрона из валентной зоны при столкновении. Тогда выбитый электрон также будет разгоняться и выбивать другие электроны, что приведет к возрастанию тока.\\
Описывается полуэмпирической формулой:
\begin{equation}
M = \frac{|I|}{I_0} = \frac{1}{1 - (\frac{U}{U_m})^n}
\end{equation}
Где M - коэффициент ионизации, $U_m$ - напряжение пробоя, n - значение, экспериментально определяемое для каждого вида полупроводника $\approx 3 \div 5$.


Сами стабилитроны делают в основном кремниевые, т.к. у них обратный ток до пробоя меньше и они не греются.\\
Для стабилитронов существенна зависимость напряжения пробоя от температуры, определяемой температурной чувствительностью: $\varepsilon = \frac{dU_{break}}{dT}$, которая отрицательная для низковольтных туннельный диодов и положительная для высоковольтных лавинных.

К параметрам стабилитрона относятся:
\begin{itemize}
\item Напряжение пробоя (стабилизации)
\item Максимальный ток стабилизации
\item Минимальный ток стабилизации
\item Дифференциальное сопротивление $r_d = \frac{d U_{st}}{d I_{st}}$
\item Температурный коэффициент 
\end{itemize}
