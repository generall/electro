\subsubsection{Почему у диодов в качестве предельных эксплуатационных параметров указывается Uобр Iпр. сред, I пр. макс}

Основной причиной, по которой диод (да и вообще любой полупроводниковый прибор) может функционировать неверно (отлично от ожидаемых аппроксимаций) - это изменение температуры функционирования.
Среднее количество теплоты, рассеиваемой телом, пропорционально его температуре, а мощность нагрева пропорциональна току помноженному на напряжение, таким образом, при заданном среднем значении тока можно вычислить среднюю температуру в данном режиме (которая зависит от кучи параметров включая геометрию тела и наличие радиатора). Этим обуславливается наличие такого параметра как максимальный средний ток.


Однако температура распространяется по телу не мгновенно, более того, если изначально тело находится в рабочей температуре, то нагрев до критической требует некоторого количества энергии, следовательно, допустимо подавать на прибор кратковременное напряжение выше среднего максимального без увеличения его температуры выше критической. Этим объясняется параметр I пиковое максимальное.


Максимальное обратное напряжение обусловлено напряжением теплового пробоя диода, которое приводит к резкому увеличению тока при маленьком изменении падения напряжения, что приводит к выделению большой мощности и нагреву прибора.

