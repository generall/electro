
\documentclass[12pt,a4paper]{article} 
\usepackage{indentfirst,latexsym,graphicx} 
\usepackage[utf8]{inputenc} % Включаем поддержку UTF8
\usepackage[russian]{babel}
\usepackage{amssymb,amsmath} 
\graphicspath{{img/}} % тут картинки твои
\usepackage[left=2cm,right=2cm,top=2cm,bottom=2cm,bindingoffset=0cm]{geometry}
\usepackage{amsmath, amssymb, amsthm}

\begin{document}
Принципиальная схема:
\begin{center}
	\begin{figure}[h!]
		\center{\includegraphics[scale=0.2]{oe.png}}
		\caption{каскад с ОЭ}	
		\label{OE}
	\end{figure}
\end{center}

Эквивалентная схема:
\begin{center}
	\begin{figure}[h!]
		\center{\includegraphics[scale=0.7]{oeekv.png}}
		\caption{эквивалентная схема с ОЭ}	
		\label{EOE}
	\end{figure}
\end{center}

1.Входное сопротивление:

$$R_\textit{вх}=\frac{U_\textit{вх}}{i_\textit{вх}}$$

$U_\textit{вх}=E_\textit{г}-i_\textit{вх}R_\textit{вх}$

1)Без $R_1 || R_2$, тогда $i_\textit{вх}=i_\textit{б}$

$U_\textit{вх}=\varphi_4-\varphi_3=i_\textit{б}r_\textit{б}+i_\textit{э}r_\textit{э}$
(контур 1 по закону Кирхгофа)

$i_\textit{э}=(B+1)i_\textit{б}$

$U_\textit{вх}=i_\textit{б}r_\textit{б}+(B+1)i\textit{б}r_\textit{э}$

$R_\textit{вх}=\frac{U_\textit{вх}}{i_\textit{вх}}=\frac{i_\textit{б}r_\textit{б}+(B+1)i_\textit{б}r_\textit{э}}{i_\textit{б}}$

$$
R_\textit{вхтроэ}=r_\textit{б} + (B+1)r_\textit{э}
$$

2)C учетом $R_1 || R_2$, получаем

$$
R_\textit{вх}=(R_1||R_2)R_\textit{вхтроэ}
$$

2. Выходное сопротивление:
 Выходное сопротивление определяется при отключении нагрущки и при нулевом входном сигнале
 
 $$R_\textit{вых}=\frac{U_\textit{хх}}{i_\textit{кз}}$$

$ U_\textit{хх}=I_\textit{к}R_\textit{к} $

$\frac{I_\textit{к}}{I_\textit{б}}=B,\Rightarrow U_\textit{хх}\approx BI_\textit{б}R_\textit{к}$

$$
I_\textit{кз}\approx BI_\textit{б},\Rightarrow R_\textit{выхоэ}=\frac{BI_\textit{б}R_\textit{к}}{BI_\textit{б}}
$$

3. Коэффициент передачи по напряжению:
$$
K_\textit{uоэ}=\frac{U_\textit{вых}}{U_\textit{вх}}=\frac{B(R_\textit{к}||R_\textit{н})}{R_\textit{вхтроэ}}
$$

4. Коэффициент передачи по току:

$$K_\textit{iоэ}=\frac{i_\textit{вых}}{i_\textit{вх}}=\frac{i_\textit{н}}{i_\textit{вх}}$$

$R_k \textit{ и } R_H$ включены параллельно $i_\textit{к}$ входит в $R_\textit{k}||R_\textit{Н}$

$$i_\textit{н}=i_\textit{к}\frac{R_\textit{к}}{R_\textit{н}+R_\textit{к}}$$

$i_\textit{вх}$ - ток через $[R_1||R_2]||R_\textit{вхтроэ}\Rightarrow$

$i_\textit{б}+i_{R_1R_2}=i_\textit{вх}$

$U_{R_1R_2}=U_\textit{вхтроэ}$

$i_{R_1R_2}[R_1||R_2]=i_\textit{б}R_\textit{вхтроэ}$

$i_{R_1R_2}=i_\textit{б}\frac{R_\textit{вхтроэ}}{R_1||R_2}$

$i_\textit{вх}=i_{R_1R_2}+i_\textit{б}=\left(\frac{R_\textit{вхтроэ}}{R_1||R_2}+1\right)i_\textit{б}$

$$K_i=i_k\frac{R_k}{R_k+R_H}\frac{1}{\left(\frac{R_\textit{вхтроэ}}{R_1||R_2}+1\right)i_\textit{б}}=B\frac{R_k}{R_k+R_H}\frac{R_1||R_2}{R_\textit{вхтроэ}+R_1||R_2}
$$


Каскад ОБ:
Принципиальная схема:
\begin{center}
	\begin{figure}[h!]
		\center{\includegraphics[scale=0.2]{ob.png}}
		\caption{каскад с ОБ}	
		\label{OB}
	\end{figure}
\end{center}
Эквивалентная схема:
\begin{center}
	\begin{figure}[h!]
		\center{\includegraphics[scale=0.9]{eob.png}}
		\caption{эквивалентная схема с ОБ}	
		\label{EOB}
	\end{figure}
\end{center}

1.Входное сопротивление:
$$R_\textit{вх}=\frac{U_\textit{вх}}{i_\textit{вх}}$$

$$U_\textit{вх}=E_\textit{г}-i_\textit{вх}R_\textit{г}$$

1)Без учета $R_\textit{э}$, тогда $i_\textit{вх}=i_\textit{э}$

$U_\textit{вх}=\varphi_2-\varphi_1=i_\textit{э}r_\textit{э}+i_\textit{б}r_\textit{б}$
(контур 1 по II-му закону Кирхгофа)
Из эквивалентной схемы:
$i_\textit{б}=(\alpha+1)i_\textit{э}$ - По первому з-ну Кирхгофа

$U_\textit{вх}=i_\textit{э}r_\textit{э}+(\alpha+1)i_\textit{э}r\textit{б}$

$R_\textit{вхтроб}=\frac{U_\textit{вх}}{i_\textit{вх}}=\frac{i_\textit{э}r_\textit{э}+(\alpha+1)i_\textit{э}r_\textit{э}}{i_\textit{э}}$

$$
R_\textit{вхтроб}=r_\textit{э} + (\alpha+1)r_\textit{б}
$$

2)C учетом $R_\textit{э}$ получаем

$$
R_\textit{вхтроб}=R_\textit{э}||R_\textit{вхтроэ}
$$

2. Выходное сопротивление:
 Выходное сопротивление определяется при отключении нагрущки и при нулевом входном сигнале
 
 $$R_\textit{вых}=\frac{U_\textit{хх}}{i_\textit{кз}}$$

$ I_\textit{кз}\approx \alpha i_\textit{э}$

$$
\frac{I_\textit{к}}{I_\textit{э}}=\alpha,\Rightarrow R_\textit{вых}=\frac{I_\textit{к}R_\textit{к}}{\alpha I_\textit{э}}=\frac{\alpha I_\textit{э}R_\textit{к}}{\alpha I_\textit{э}}\approx R_\textit{к}
$$

3. Коэффициент передачи по напряжению:

$$
K_\textit{uоб}=\frac{U_\textit{вых}}{U_\textit{вх}}=\frac{B(R_\textit{к}||R_\textit{н})}{R_\textit{вхтроб}}
$$

4. Коэффициент передачи по току:

$$K_\textit{iоэ}=\frac{i_\textit{вых}}{i_\textit{вх}}=\frac{i_\textit{н}}{i_\textit{вх}}$$

$i_H=i_K\frac{R_k}{R_H+R_k}$

$i_{R_\textit{э}}=i_\textit{э}\frac{R_\textit{вхтроб}}{R_\textit{Э}}$

$$
K_{I_\textit{ОБ}}=i_k\frac{R_k}{R_k+R_H}\frac{1}{\frac{R_\textit{вхтроб}+R_\textit{Э}}{R_\textit{Э}}i_\textit{э}}=\alpha \frac{R_k}{R_k+R_H} \frac{R_\textit{Э}}{R_\textit{вхтроб}+R_\textit{э}}
$$


\end{document}