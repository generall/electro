\subsubsection{Уравнение непрерывности, какие процессы в полупроводнике описывает это уравнение?}
Временные изменение концентрации носителей в проводнике обуславливается следующими процессами:
\begin{itemize}
\item Генерация свободных носителей заряда под действием внешних факторов. $\Delta g_n , \Delta g_p$ - скорости генерации электронов и дырок соответственно.
\item Рекомбинация избыточных носителей. $\frac{n - n_0}{\tau_n}$ - скорость рекомбинации
\item Приток свободных носителей в область dx. Тогда $\frac{1}{q}\nabla \cdot j_{n}$ - разность между втекающими и вытекающими в обрасти dx для одномерного случая.
\end{itemize}

Получаем следующее уравнение:
\begin{equation}
\frac{\partial n}{ \partial t} = \Delta g_n - \frac{n - n_0}{\tau_n} + \frac{1}{q} \nabla \cdot j_{n}
\end{equation}
Для p аналогично, но ток должен быть взят со знаком минус т.к. при увеличении тока с координатой накапливание заряда уменьшается, что видно из следующего уравнения:
\begin{equation}
div(\frac{j}{q}) = \frac{1}{q} \nabla \cdot j = \frac{1}{q}\frac{\partial}{ \partial x} \left[ j_{dif} + j_{flow} \right]
\end{equation}

Подставляя выражения для токов из уравнения переноса в уравнение для дивергенции, дифференцируя и подставляя в уравнение непрерывности получаем следующее уравнение:
\begin{equation}
\frac{\partial n}{ \partial t} = -  \frac{n - n_0}{\tau_n}  + D_n \frac{\partial^2 n }{x^2} + \mu_n E \frac{\partial n}{\partial x} + n \mu_n \frac{\partial E}{\partial x}
\end{equation}
Сие и есть уравнение непрерывности, являющееся в общем случае следствием закона сохранения заряда. Аналогичное уравнение можно получить для p.\\

Если предположить, что в полупроводнике отсутствует электрическое поле $E=0$, то уравнение непрерывности упрощается до уравнения диффузии:
\begin{equation}
\frac{\partial n}{ \partial t} = -  \frac{n - n_0}{\tau_n}  + D_n \frac{\partial^2 n }{x^2}
\end{equation}
Если же полем пренебрегать нельзя, то необходимо применять\textit{ уравнение Пуассона}, определяющее изменение электрического поля с координатой в среде:
\begin{equation}
\frac{\partial E}{\partial x} = \frac{\lambda}{\varepsilon \varepsilon_0}
\end{equation}

Поскольку в условиях нейтральности $\lambda = 0$, можно считать, что объемный заряд есть следствие приращения концентраций, т.е. можно выразить через избыточные концентрации:  $\lambda = q (\Delta p - \Delta n)$ - плотность заряда



\pagebreak
