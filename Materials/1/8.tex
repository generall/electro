\subsubsection{Чем определяется инерционность pn-перехода? Зависит ли она от режима работы?}
np-переход является инерционным элементом по отношению к быстрым изменениям тока или напряжения т.к. для перехода в новый режим работы требуется установления нового распределения неосновных носителей заряда, а оно устанавливается не сразу. Более того, изменение напряжения на переходе изменяет его ширину, а следовательно, и величину пространственного заряда в этом переходе (см. \S \ref{sec:pn_length}).


Таким образом, для отражения инерционных процессов в переходе можно считать, что к нему подключены параллельно емкости. Их принято разделять на \textbf{барьерную}, отвечающую за перераспределение зарядов в переходе, и \textbf{диффузионную}, отражающую перераспределение зарядов в базе.
Такое разделение условно, но удобно в первую очередь тем, что соотношения этих емкостей сильно меняется в зависимости от прямого или обратного смещения перехода.


Использование эквивалентных емкостей, однако, целесообразно только при малых сигналах, когда емкости можно считать линейными, при больших сигналах использование емкости нецелесообразно и в этом случае применяют уравнение непрерывности.

\textbf{Барьерная емкость}
Рассмотрим асимметричный переход, практически полностью расположенный в базе типа $n$, т.е. $l = l_n$\\
Под барьерной емкостью будем понимать то количество зарядов, которые скопились в переходе. Одна следует отметить, что в переходе нет свободных носителей заряда, а есть только заряд, обусловленный примесью ($N_a; N_d$). Таким образом, основная инертность состоит в изгнании лишнего заряда из перехода при его расширении и внесении заряда в освободившуюся область при его сужении. Т.о. можно говорить о заряде конденсатора как о заряде расположенном в области, свободной от всяких распределений и определяемой только концентрацией $N_d$ (в случае n-базы), а инертность будет привнесена уравнением длины перехода:
\begin{equation}
l \approx l_0 \sqrt{\frac{|U|}{\Delta \phi_0}}
\end{equation}

Заряд тогда:
\begin{equation}
Q_n = q N_d (S l)
\end{equation}
Где $S l = V$ - объем\\

Подставляя в это уравнение выражение длины перехода и дифференцируя по напряжению получаем выражение для диф. емкости:

\begin{equation}
C = \frac{dQ}{dU} = \frac{S \sqrt{0.5 \varepsilon \varepsilon_0 q N_d}}{\sqrt{|U|}} = \frac{ \varepsilon \varepsilon_0 S}{l_0} \sqrt{\frac{\Delta \phi_0}{|U|}}
\end{equation}
Или в более общем случае без предположения, что $U >> \Delta \phi_0$ :
\begin{equation}
C = \frac{ \varepsilon \varepsilon_0 S}{l_0} \sqrt{\frac{\Delta \phi_0}{\Delta \phi_0 - U}}
\end{equation}

Приблизительная величина этой емкости при $l_0 = 0.5 * 10^{-6} m; \Delta \phi_0 = 0.75 V; S = 0.01 sm^2; U = 20 V  \longrightarrow C_b = 50*10^{-12} F$



\textbf{Диффузионная емкость}


Диффузионная емкость обусловлена накоплением зарядов в базовом слое из-за инжекции.\\
Зная распределение инжектированных зарядов в базе (которое должно выводиться в тепловом токе), а именно:
\begin{equation}
\Delta p (x) = p_0 (e^{\frac{U}{\phi_t}}-1) \frac{\sh(\frac{\omega - x}{L})}{\sh(\frac{\omega}{L})}
\end{equation}
Можем получить заряд обкладки:
\begin{equation}
\Delta Q = q S \int_{0}^{\omega} \Delta p(x) dx = \frac{q S L \Delta p (0)}{\th(\omega / L)}(1 - sech \frac{\omega}{L})
\end{equation}
И учитывая выражение $L = \sqrt{D \tau}$\\
Подставим в данное выражение значение $\Delta p (0) = p_0 \frac{I}{I_0}$, определенное через ток и выражение для самого тока $I_0 \approx q \frac{DS}{L \th(\frac{\omega}{L})} p_0$, получаем:

\begin{equation}
\Delta Q = I \tau (1 - sech \frac{\omega}{L})
\end{equation}
Интегральная емкость:
\begin{equation}
C_I = \frac{\Delta Q}{U} = \frac{\tau}{R_d}  (1 - sech \frac{\omega}{L})
\end{equation}
Где $R_d = U/I$ - сопротивление диода постоянному току.\\
Дифференциальная:
\begin{equation}
C_d = \frac{d (\Delta Q)}{dU} =  \frac{\tau}{r_d}  (1 - sech \frac{\omega}{L}) =  \frac{ I \tau}{\phi_t}  (1 - sech \frac{\omega}{L})
\end{equation}

Для толстой базы:
\begin{eqnarray}
\Delta Q \approx I \tau \\
C_d \approx \frac{I \tau}{\phi_t}
\end{eqnarray}

Для тонкой базы выражение упрощается до:
\begin{eqnarray}
\Delta Q \approx I \frac{\omega^2}{2 D}\\
C_d \approx \frac{I  \frac{\omega^2}{2 D}}{\phi_t}
\end{eqnarray}

Приблизительные значения дифференциальной емкости при $\tau = 5*10^{-6}s ; I = 0.01 A \longrightarrow C_d = 2 *10^{-6} F$

\pagebreak
