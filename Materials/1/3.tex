\subsubsection{Уравнение переноса. Какую информацию содержит это уравнение?}

В общем случае движение носителей заряда в полупроводниках обусловлено двумя процессами: диффузией под действием градиента концентрации и дрейфом под действием градиента потенциала в электрическом поле. Поскольку в полупроводниках мы имеем дело с двумя типами носителей - дырками и электронами, полный ток состоит из 4 составляющих(воспользуемся уравнением для плотности):
$$
j = (j_p)_{dif} + (j_p)_{dr} + (j_n)_{dif} + (j_n)_{dr} 
$$

Плотности дрейфовых составляющих тока пропорциональны градиенту электрического потенциала $\phi$, т.е. напряженности электрического поля E. В одномерном случае, когда движение носителей происходит только вдоль оси $x$:
$$
(j_p)_{dr} = -qp\mu_p\frac{\delta \varphi}{\delta x} = qp\mu_pE
$$
$$
(j_n)_{dr} = -qn\mu_n\frac{\delta \varphi}{\delta x} = qn\mu_nE
$$


Плотности диффузионных составляющих тока:
$$
(j_p)_{dif} = -qD_p\frac{dp}{dx}
$$
$$
(j_n)_{dif} = qD_n\frac{dn}{dx}
$$

Диффузионный ток образуется в результате разной степени лигированности п/п-ка примесями, т.е. когда заряды распределены неравномерно. Ток направлен в сторону убывания концентрации.
%У Сони еще написано про рекомбинацию, но если писать про нее, надо также упомянуть еще и про генерацию.

Здесь $D_p$ и $D_n$ - коэффициенты диффузии дырок и электронов. Связаны с подвижностями тех же носителей формулой Эйнштейна:
$$
D = \varphi_t\mu
$$

Получим плотность полного тока:
$$
j = -qD_p\frac{dp}{dx} + qp\mu_pE + qD_n\frac{dn}{dx} + qn\mu_nE
$$
\pagebreak
